\documentclass{article}
\usepackage[utf8]{inputenc}
\usepackage[portuguese]{babel}

\usepackage{listings}
\usepackage{color}
 
\definecolor{codegreen}{rgb}{0,0.6,0}
\definecolor{codegray}{rgb}{0.5,0.5,0.5}
\definecolor{codepurple}{rgb}{0.58,0,0.82}
\definecolor{backcolour}{rgb}{0.95,0.95,0.92}
 
\lstdefinestyle{slstyle}{
    backgroundcolor=\color{backcolour},   
    commentstyle=\color{codegreen},
    keywordstyle=\color{magenta},
    numberstyle=\tiny\color{codegray},
    stringstyle=\color{codepurple},
    basicstyle=\footnotesize,
    breakatwhitespace=false,         
    breaklines=true,                 
    captionpos=b,                    
    keepspaces=true,                 
    numbers=left,                    
    numbersep=5pt,                  
    showspaces=false,                
    showstringspaces=false,
    showtabs=false,                  
    tabsize=2
}
 
\lstset{style=slstyle}

\title{Especificação da Linguagem}
\author{Grupo 12, LFA}
\date{Maio 2018}

\usepackage{natbib}
\usepackage{graphicx}

\begin{document}

\maketitle

\tableofcontents
\vspace{2cm} %Add a 2cm space

\section{Estruturas de Dados (Tipos de Variáveis)} \label{variables}
\subsection{Sequências}
Uma sequência de notas e silêncios pode ser definida, através da palavra chave \texttt{sequence}, como:
\begin{lstlisting}[language=C]
sequence melody = [C C G G A A G R F F E E D D C R]; // os espacos sao opcionais
\end{lstlisting}

\subsubsection{Notas musicais}
A notas são identificadas pela sua letra:
\begin{itemize}
    \item \texttt{C} para Dó,
    \item \texttt{D} para Ré,
    \item \texttt{E} para Mi,
    \item \texttt{F} para Fá,
    \item \texttt{G} para Sol,
    \item \texttt{A} para Lá,
    \item \texttt{B} para Si.
\end{itemize}
A letra \texttt{R} é usada para silêncios (derivada de \textit{Rests}).

Um cardinal (\texttt{\#}) a seguir à letra sobe a respetiva nota por meio tom; um b minúsculo (\texttt{b}) a seguir à letra desce a respetiva nota por meio tom.

% Limite de # e b? (i.e. so 1 por nota?)

Pode, depois do tom, aparecer um número, entre 1 e 8. Este especifica a oitava a usar, sendo 1 a mais grave e 8 a mais aguda. Por omissão, a quarta oitava é usada.

Assim, a sequência já apresentada podia reescrita da seguinte forma:
\begin{lstlisting}[language=C]
sequence melody = [C4 B# G G4 A4 A G R F E#4 Fb4 E D D4 C4 R];
\end{lstlisting}
% Outro exemplo a variar oitavas?

\subsubsection{Duração duma nota}
Por omissão, uma nota demora um tempo\footnote{Em termos musicais, uma semínima.}. A sua duração depende do \textit{tempo}, ou \texttt{BPM}\footnote{\textit{Beats Per Minute}}, da música. Esta configuração é explorada na secção \ref{config}.

Uma nota pode, no entanto, tocar mais ou menos tempo. Há duas formas de especificar a duração duma dada nota:
\begin{enumerate}
    \item \textbf{Por extensão.} A duração da nota é especificada através de chavetas (\texttt{\{} e \texttt{\}}), relativamente à duração unitária. Por exemplo, \texttt{C\{4\}} demora o quádruplo do tempo de \texttt{C}, e \texttt{C\{0.75\}} demora três quartos do tempo de \texttt{C}.
    \item \textbf{Notação simplificada.} Utilizam-se apóstrofos para reduzir a duração duma nota em metade, havendo uma correspondência direta com a duração das notas musicais convencionadas
    \footnote{Semínima (duração de 1), colcheia (duração de 1/2), semicolcheia (duração de 1/4), etc.}. 
    Por exemplo, \texttt{C'} demora metade do tempo de \texttt{C}, e \texttt{C''} apenas um quarto. 
    % suportar aumentar tempos em notacao simplificada?
    % suportar ligaduras?
\end{enumerate}

% Ligaduras (prolongamento de som)
% Se duas notas estão ligadas por um underscore (\texttt{_}), então considera-se que as duas notas estão ligadas, ou seja, são tocadas como uma única nota contiínua.
% http://www.thenewdrummer.com/read-drum-music-pt7-tied-notes/

% suportar Slurs (tocar a transicao entre 2 notas o mais smoothly possivel)?

\subsubsection{Acordes}
O símbolo \texttt{|} é usado para tocar várias notas em simultâneo, numa só sequência.
Para tocar o acorde de Dó maior (C, E e G), na quarta oitava, podíamos então escrever:
\begin{lstlisting}[language=C]
sequence intro = [C|E|G];
\end{lstlisting}

\subsection{Performances}
A associação duma sequência musical com um instrumento representa um terceiro tipo de dados, uma \texttt{performance}:
\begin{lstlisting}[language=C]
sequence twinkle = [CC GG AA G{2} FF EE DD C{2}];
performance p = twinkle on guitar;

// ou, alternativamente, definindo a sequencia implicitamente
performance p = [CC GG AA G{2} FF EE DD C{2}] on guitar;
\end{lstlisting}

\subsection{Marcos de Tempo}
Um marco de tempo é uma variável que referencia um dado momento entre o início e o fim, inclusivé, da peça musical. \texttt{start} é um marco de tempo pré-definido, representando o início da peça. 
\subsubsection{Funções auxiliares}
A função auxiliar \texttt{duration}, aceita um parâmetro, do tipo sequence ou performance, e devolve um time igual à sua duração.
\begin{lstlisting}[language=C]
sequence intro = [R{4} C{4} G{4} C5{3.5} E|G|C5{.5} Eb|G|C5{8} C{4} G{4}]; // Strauss - Also Sprach Zarathustra - Intro (https://www.8notes.com/scores/7213.asp)
time endIntro = start + duration(intro);
\end{lstlisting}

\subsection{Inteiros}
Inteiros também são suportados:
\begin{lstlisting}[language=C]
int octave = 4;
\end{lstlisting}

\subsection{Arrays (TODO)}
Um array é uma coleção de várias instâncias da mesma estrutura de dados. Suportam-se arrays de sequências, instrumentos e performances.

\section{Geração de aúdio} \label{audio}
\subsection{Instrumentos}
Para tocar sequência musical é, naturalmente, necessário especificar que instrumento deve ser utilizado. Assim, para tocar \textit{Twinkle, Twinkle, Little Star} com uma piano, teríamos:
\begin{lstlisting}[language=C]
// definir a sequencia
sequence twinkle =  [CC GG AA G{2} FF EE DD C{2}];

// utilizando performances
performance p = twinkle on piano;
play p;

// ou, alternativamente, definindo a performance implicitamente
play twinkle on piano;

// ou definindo a performance e a sequencia implicitamente
play [CC GG AA G{2} FF EE DD C{2}] on piano;
\end{lstlisting}

Vários instrumentos podem ser usados para tocar uma dada sequência. No entantom nem todos suportam o mesmo registo (por exemplo, um piano suporta uma maior gama de notas que um violino). Se a um instrumento é dada uma sequência que este não suporta, as notas não suportadas são \textit{aparadas}, sendo substituídas pela nota mais próxima que é suportada. 

Os instrumentos disponíveis são os seguintes:
\begin{itemize}
    \item \texttt{piano};
    \item \texttt{guitar};
    \item \texttt{violin};
    \item \texttt{cello};
    \item \texttt{bass};
    \item \texttt{drums}.
\end{itemize}
A criação de novos instrumentos é suportada, estando detalhada na secção \ref{config}.

\subsection{Modos de reprodução}
No exemplo anterior, não foi especificado quando começar a tocar a sequência. Por omissão, a sequência começa a ser tocada no início da peça (por outras palavras, no tempo 0). Este tempo também pode ser obtido através da palavra chave \texttt{start}.

Averiguemos os diferentes modos de reprodução:

\subsubsection{Simultâneo}
Por omissão, todas as sequências são tocadas começando no tempo 0, ou \texttt{start}. Se há mais que uma sequência a ser tocada, todas as sequências são tocadas em paralelo.

\begin{lstlisting}[language=C]
play [CC GG AA G{2} FF EE DD C{2}] on piano;

// e equivalente, em termos do som produzido no ficheiro final, a
play [CC RR RR G{2} FR RE DR C{2}] on piano;
play [RR GG AA RR RF ER RD RR] on piano;
\end{lstlisting}

Um outro exemplo de reprodução simultânea é o seguinte, que separa melodia e harmonia em duas performances diferentes:
\begin{lstlisting}[language=C]
// definir sequencias
sequence twinkle = [CC GG AA G{2} FF EE DD C{2}];
sequence twinkle_bass = [C|E|G{2} F|A|C{2} C|E|G{4} F|A|C C|E|G G|B|D C|E|G];

// tocar performances
play twinkle on guitar;
play twinkle_bass on guitar;
\end{lstlisting}

\subsubsection{Sequencial}
A palavra chave \texttt{after} indica que uma dada performance deve começar imediatamente após o fim doutra.
\begin{lstlisting}[language=C]
// definir sequencias
sequence first_line = [CC GG AA G{2}];
sequence second_line = [FF EE DD C{2}];

// definir performances
performance first_line = twinkle on guitar;
performance second_line = twinkle_bass on guitar;

// tocar performances
play first_line;
after first_line play second_line;
\end{lstlisting}

Num exemplo mais avançado, onde a sequência de referência (no exemplo acima, \texttt{first\_line}) passada a after é tocada mais que uma vez, pode ser especificado após que performances deve a sequência alvo (no exemplo acima, \texttt{second\_line}) ser tocada. Por omissão, a sequência alvo é tocada apenas 1 vez, após a primeira reprodução da sequência de referência. 

Para obter outros comportamentos, a palavra chave \texttt{always} pode ser utilizada.

\begin{lstlisting}[language=C]
// definir sequencias
sequence first_line = [CC GG AA G{2}];
sequence second_line = [FF EE DD C{2}];

// definir performances
performance first_line = twinkle on guitar;
performance second_line = twinkle_bass on guitar;

// tocar performances
play first_line;
after first_line always play second_line;
after second_line play first_line;
// toca first_line (FL), seguido de second_line (SL), e repete uma vez, ou seja, FL, SL, FL, SL
\end{lstlisting}

% Permitir tocar apos momentos especificos?
% i.e. apos first_line ser tocado pela segunda e terceira vezes, toca second_line

\subsubsection{Usando Marcos de Tempo}
A palavra chave \texttt{at} específica um Marco de Tempo específico no qual a performance deve começar, independentemente de haver outras performances a decorrer nesse momento. 
\begin{lstlisting}[language=C]
performance verse = [CC E{2} GG B{2} C5C5 GG C{4}] on violin;
performance chorus = [GAGA ABBA GEGE EBBE] on violin;

time chorusStart = start + 2*duration(verse);

// tocar performances
play verse;
play verse;
at chorusStart play chorus;
\end{lstlisting}

Uma performance pode ser tocada em mais que um momento. Para isso, além da palavra chave \texttt{at}, utiliza-se também a palavra chave  \texttt{and}.
\begin{lstlisting}[language=C]
performance verse = [CC E{2} GG B{2} C5C5 GG C{4}] on violin;
performance chorus = [GAGA ABBA GEGE EBBE] on violin;

time chorusStart = start + 2*duration(verse);
time otherTimeChorusStarts = start + 3.14*duration(verse);

// tocar performances (chorus e tocada 2 vezes)
play verse;
at chorusStart and otherTimeChorusStarts play chorus;
\end{lstlisting}

\subsubsection{Repetição}
Há duas palavras chaves que permitem a repetição: o uso da palavra chave \texttt{times} permite repetir uma performance 0 ou mais vezes. \texttt{loop} permite repetir uma performance até ao fim da peça  (\texttt{loop} = \texttt{play} $\infty$ \texttt{times}).
\begin{lstlisting}[language=C]
performance verse = [CC E{2} GG B{2} C5C5 GG C{4}] on violin;
performance chorus = [GAGA ABBA GEGE EBBE] on violin;
performance bass = [G|B|D{4} A|D|F#{4} G|B|D{4} A|D|F#{4}] on bass;

time chorusStart = start + 2*duration(verse);

// tocar performances
play verse 2 times on piano;
at chorusStart play chorus;

loop bass; // repeta ate ao fim da musica
\end{lstlisting}

\subsection{Uso de arrays (TODO)}
Palavras chaves all e sequentially
\begin{lstlisting}[language=C]
// perform
play sequentially all melody_lines // , back-to-back, consecutively
repeat_times times on piano;


(AFTER _ (ALWAYS?) | AT _ (AND _)* )? PLAY (SEQ? ALL)? _ (INT TIMES)? ON INST;

\end{lstlisting}

\subsection{Modulações (TODO)}
\subsubsection{de Tom}
Permitir slurring (mudança dinâmica) ou só pitch change (mudança "estática")? (Depende da biblioteca usada no backend)
\subsubsection{de Tempo}
Acelerar ou desacelarar o BPM da muúsica, temporariamente ou nao

\section{Interação com o exterior} \label{exterior}
\subsection{Estruturas de dados auxiliares}
\subsubsection{Strings}
Strings são sequências de caracteres, números e símbolos delimitadas por aspas (\texttt{"}). Dentro duma string, aspas podem ser escapadas através de \texttt{\textbackslash"}.

Não existe um tipo de dados String explícito, sendo este usado apenas como paraâmetro opcional para funções de I/O.
\subsection{getInt( string? )}
\texttt{getInt()} permite obter um inteiro através do \textit{Standard In}. 

Opcionalmente, pode ser passada uma String, que será impressa antes de aguardar a resposta do utilizador (uma String de \textit{prompt}).

\subsection{getSequence( string? )}
À semelhança de \texttt{getInt()}, \texttt{getSequence()} permite obter uma sequência através do \textit{Standard In}.

Opcionalmente, pode ser passada uma String, que será impressa antes de aguardar a resposta do utilizador (uma String de \textit{prompt}).

% \subsection{putString(string)}
% Imprime a string dada através do \textit{Standard Out}. Strings são sequências de caracteres, números e símbolos delimitadas por aspas (\texttt{"}). Dentro duma string, aspas podem ser escapadas através de \texttt{\textbackslash"}.

% Não existe um tipo de dados String explícito, sendo este usado para esta função apenas. % mudar?

% File IO?

\section{Controlo de fluxo (TODO)} \label{flux}
\subsection{Instruções condicionais}
\subsubsection{if}
\begin{lstlisting}[language=C]
if :
\end{lstlisting}

\subsubsection{with ... choose}
\begin{lstlisting}[language=C]
with getInt() choose:
    case 1:
        play seq on violin;
    case 2:
        play seq on guitar;
    default:
        play seq on piano;
\end{lstlisting}

\subsection{Intruções de repetição}
loops - Faz sentido ter loops? Talvez forEach de instrumentos?

Ter arrays? De sequencias, instrumentos(?), times e performances?


\section{Configurações (TODO)} \label{config}
\subsection{BPM (Beats Per Minute)}
\subsection{Especificação de (novos) instrumentos}
(Depende da biblioteca externa...)


\section{Exemplos (TODO)} \label{example}
\subsection{Parabéns}
\begin{lstlisting}[language=C]
// Happy Birthday
// override BPM setting
// BPM = 160;

// have time signature?

// get user input
int repeat_times = getInt("Number of repetitions: ");

// define melody sequences
sequence[] melody_lines = [
    [D{1.5} D{0.5}   E  D G F#{2}], 
    [D{1.5} D{0.5}   E  D A G{2}],
    [D{1.5} D{0.5}   D5 B G F# E],
    [C5{1.5} C5{0.5} B  G A G{3}]];

// perform
play sequentially all melody_lines repeat_times times on piano;
\end{lstlisting}

\subsection{Looping machine like thingy?}
\begin{lstlisting}[language=C]
// tipo, add looping track, add looping track, etc.
\end{lstlisting}

\subsection{Mais exemplos}
\begin{lstlisting}[language=C]
\end{lstlisting}

% \bibliographystyle{plain}
% \bibliography{references}
\end{document}
